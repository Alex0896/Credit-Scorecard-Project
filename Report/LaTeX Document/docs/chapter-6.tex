\chapter{Conclusion} \label{cha:chapter-6}

Credit scores are something everyone will have an interaction with at some point in their lives and an important piece of personal finance. Scorecard modelling is a vital tool for financial insitutions to assist in evaluating their applicants. In this project we covered some of the popular methods used in credit scoring such as WOE and IV in Section (\ref{sec:woe_and_iv}) along with performance evaluation in Section (\ref{sec:perf_eval}). In Chapter 3 we explored a data set of home equity loans and the effect of some common variables have on the bad rate of applicants such as the number of times a client has been delinquent or the number of credit lines open. Using logisitc regression we were able to develop a selection scorecard for the dataset and evaluate the performance against each other. If we were to do anything different in this project, considering methods other than logistic regression such as support vector machines or decision trees would be our first choice, although logistic regression is the most popular for credit scoring due to its simplicity and reliabiltiy, there are these other methods which challenge its position. \\

We then demonstrated areas other than finance in which scorecard modelling and methods can be applied by using an open data set of covid-19 patients. Although the data set does not represet a true demographic for covid patients, it did allow us to see the potential use of scorecards to assess a patients risk. 