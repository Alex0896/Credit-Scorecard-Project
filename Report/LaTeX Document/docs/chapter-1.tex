\chapter{Credit Scoring}

\section{Introduction}
Credit scoring is a method used by financial institutions globally to assess whether a customer should be accepted and taken on. This can be for a variety of services such as credit cards, loans, mortgages, etc. Its development originated from the need of minmizing risk and maximizing rewards. Lenders needed a way of determining if a potential customer would be able to pay back their credit and avoid delinqency [\ref{sec:glossary}] and as such not costing the lender money by taking on creditees which end up being unable to repay the debt. A credit score is usually just a number indicating your quality as a creditee. The scale of the score can vary on which lender is providing the score but usual ones are 0-999 or 0-500 with the lower the score the less likely you would be offered the service. The number typically represents your chance of defaulting [\ref{sec:glossary}]. \\

Although used globally, there is no widely accepted ``perfect'' model or method. All companies assess their customers differently, a customer could be rejected from one lender and be accepted by another based on what they would define as an acceptable client for them. Even within companies the models and methods can change due to new circumstances and the changing financial climate. What previously could have been a strong predictor of a bad client could now be insignifcant. Changes like this require lenders to be constantly revaluating how they assess their customers to reduce the rejection of good clients and the accepting of bad ones.

\section{Home Equity Loans}
The data set which will be used in this project is a set of observations of home equity loans in the US. Home equity can total to be a large portion of a US households wealth but unlike other wealth, it cannot be liquidated easily. Dispite its inability to be liquidated easily, it is still a readily accepted form of collateral for credit \parencite{canner1998recent} with \parencite{weicher1997home} estimating it to account for 5-10\% of the US's total mortagage origination in the 1990s. From more recent data, home equity loan origination [\ref{sec:glossary}] peaked in 2005 totalling nearly \$364 billion but declined during the 2008 market crash and has since, in recent years seen a steady increase \parencite{corelogic2016home}. \\

 Before the crisis home equity loans were popular to be piggybacked [\ref{sec:glossary}] onto first mortgages with subprime rates [\ref{sec:glossary}]. In recent years since the market crash companys have been more conservative when it comes to the acceptance of applicants which brings about the importance of robust credit score modelling \parencite{corelogic2016home}.

\section{Modelling}
Credit score modelling is often discrete based, with the usual being wether a client will default or not, the most common being a logistic regression with the response variable being either a good [\ref{sec:glossary}] $(y=0)$ or a bad [\ref{sec:glossary}] customer $(y=1)$. Predictors can be a variety of variables such as personal characteristics, age, gender or economic status e.g. car owner, home owner/rentor etc, to financial characteristics like amont of current debt and repayment statuses. One thing to note, certain personal characteristics are off limit to company due to discrimination laws. Predictors such as race may be shown to have some use in scoring but cannot be used as the model would become discriminatory.\\

\section{Outline}
In this project, first we will discuss current literature on credit score modelling and other methods for data cleaning and preperation that we will be using throughout our project in Chapter (\ref{chapter:2}). Next we will describe the data used for the credit scoring and how we decided to prepare it, following from that we present some exploratory data analysis of each variable and our assumptions of their significance and characteritics in Chapter (\ref{chapter:3}). We will be using weight of evidence, explained in Section (\ref{sec:woe_and_iv}), to create binned variables. In Chapter (\ref{chapter:4}) we will cover the results of the modelling methods applied and their comparisons and performance. Finally, in Chapter (\ref{chapter:5}) we will be covering an alternative use of scorecard modelling using the example of Covid-19 and developing a model for the health risk of a patient.